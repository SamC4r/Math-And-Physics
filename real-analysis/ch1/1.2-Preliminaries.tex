\documentclass[a4paper]{article}

\usepackage[utf8]{inputenc}
\usepackage[T1]{fontenc}
\usepackage{textcomp}
\usepackage{amsmath, amssymb}

% figure support
\usepackage{import}
\usepackage{xifthen}
\pdfminorversion=7
\usepackage{pdfpages}
\usepackage{transparent}
\newcommand{\incfig}[1]{%
    \def\svgwidth{\columnwidth}
    \import{./figures/}{#1.pdf_tex}
}

\pdfsuppresswarningpagegroup=1

\title{Preliminaries}
\author{Samuel Caraballo Chichiraldi}
\date{\today}

\begin{document}
    \maketitle

    \section*{Sets}
    Collection of objects called elements.  We write $x \in A$ if $x$ is an element of $A$.  \\


    \textbf{De Morgan's Law}
    \begin{itemize}
        
    \item \[ \overline{A \cup B} = \overline{A} \cap \overline{B} \] 

    \item \[ \overline{A \cap B} = \overline{A} \cup \overline{B} \] 
    \end{itemize}


    \section*{Functions}

\textbf{Definition}: Given two sets $A$ and $B$, a function from $A$ to $B$ is a rule that takes each element $x \in A$ and associates with it a single element of $B$.  In that case we write: 
    \begin{align*}
        f: A &\longrightarrow B \\
    \end{align*}

    Given an element $x$, the expression $f(x)$ is used to represent the element of $B$ associated with $x$ by $f$. 
    The set $A$ is called the \textit{\textbf{domain}} of $f$. The \textit{\textbf{range}} of $f$ refers to the subset of $B$ given by $\{y \in B  :  y = f(x) \text{ for some } x \in A\} $

    Dirichlet’s definition of function liberates the term from its interpretation as a type of “formula.”

\end{document}
