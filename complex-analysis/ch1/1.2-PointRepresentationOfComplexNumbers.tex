\documentclass{article}

\usepackage[utf8]{inputenc}
\usepackage{amsmath, amsfonts, amssymb}

\title{Point Representation in the Complex Plane}
\author{Samuel Caraballo Chichiraldi}
\date{\today}

\begin{document}

\maketitle

\section*{The Complex Plane}

When using geometry to represent complex numbers, we place them on a two-dimensional coordinate system known as the \textbf{complex plane}, or the \textbf{$z$-plane}.

\begin{itemize}
    \item The horizontal axis (real axis) represents real numbers.
    \item The vertical axis (imaginary axis) represents imaginary numbers.
\end{itemize}

A complex number \( z = a + bi \), where \( a, b \in \mathbb{R} \), is associated with the point \( (a, b) \) in this plane. This point is often simply denoted by \( z \).

\section*{Absolute Value (Modulus)}

The \textbf{absolute value} or \textbf{modulus} of a complex number \( z = a + bi \) is its distance from the origin \( (0, 0) \) in the complex plane. By the Pythagorean Theorem:
\begin{equation}
|z| = \sqrt{a^2 + b^2}
\label{eq:modulus}
\end{equation}

\textbf{Example:}
\[
|3 - 4i| = \sqrt{3^2 + (-4)^2} = \sqrt{9 + 16} = \sqrt{25} = 5
\]

\textbf{Note:} The modulus \( |z| \) is always a non-negative real number.

\subsection*{Distance Between Two Complex Numbers}

Given two complex numbers \( z_1 = a_1 + b_1i \) and \( z_2 = a_2 + b_2i \), the distance between them in the complex plane is:
\begin{equation}
|z_1 - z_2| = \sqrt{(a_1 - a_2)^2 + (b_1 - b_2)^2}
\label{eq:distance}
\end{equation}

\section*{Geometric Interpretation}

The geometric representation of modulus is useful for describing loci (sets of points) in the complex plane.

\subsection*{Circles in the Complex Plane}

The set of all complex numbers \( z \) satisfying the equation
\begin{equation}
|z - z_0| = r
\label{eq:circle}
\end{equation}
represents a \textbf{circle} of radius \( r \) centered at \( z_0 \in \mathbb{C} \). This includes all points whose distance from \( z_0 \) is exactly \( r \).

\textbf{Example:}
Consider the equation:
\[
|z + 2| = |z - 1|
\]
This means the point \( z \) is equidistant from \( -2 \) and \( 1 \) on the real axis. Let \( z = x + iy \), then:
\[
|x + iy + 2| = |x + iy - 1| \\
\Rightarrow (x + 2)^2 + y^2 = (x - 1)^2 + y^2 \\
\Rightarrow x + 2 = -x + 1 \\
\Rightarrow x = -\frac{1}{2}
\]
This describes a vertical line \( x = -\frac{1}{2} \) in the complex plane.

\section*{Complex Conjugate}

The \textbf{complex conjugate} of a complex number \( z = a + bi \) is defined as:
\begin{equation}
\overline{z} = a - bi
\label{eq:conjugate}
\end{equation}

\textbf{Example:}
\[
\overline{-1 + 2i} = -1 - 2i
\]

\textbf{Properties:}
\begin{itemize}
    \item If \( z \in \mathbb{R} \), then \( \overline{z} = z \).
    \item Conjugation distributes over addition and subtraction:
    \begin{equation}
    \overline{z_1 + z_2} = \overline{z_1} + \overline{z_2}, \quad 
    \overline{z_1 - z_2} = \overline{z_1} - \overline{z_2}
    \label{eq:conj_add_sub}
    \end{equation}
    \item Conjugation distributes over multiplication and division:
    \begin{equation}
    \overline{z_1 z_2} = \overline{z_1} \cdot \overline{z_2}, \quad
    \overline{\left(\frac{z_1}{z_2}\right)} = \frac{\overline{z_1}}{\overline{z_2}} \quad (z_2 \ne 0)
    \label{eq:conj_mult_div}
    \end{equation}
\end{itemize}

\subsection*{Extracting Real and Imaginary Parts}

For any complex number \( z \), the real and imaginary parts can be written as:
\begin{equation}
\operatorname{Re}(z) = \frac{z + \overline{z}}{2}, \qquad 
\operatorname{Im}(z) = \frac{z - \overline{z}}{2i}
\label{eq:re_im}
\end{equation}

\section*{Product with the Conjugate}

Multiplying a complex number by its conjugate yields:
\begin{equation}
z \cdot \overline{z} = |z|^2
\label{eq:product_conjugate}
\end{equation}

\section*{Rationalizing with the Conjugate}

To divide complex numbers, especially when the denominator is complex, we use the conjugate to rationalize:

\begin{equation}
\frac{z_1}{z_2} = \frac{z_1 \cdot \overline{z}_2}{z_2 \cdot \overline{z}_2} = \frac{z_1 \cdot \overline{z}_2}{|z_2|^2}, \quad z_2 \ne 0
\label{eq:division}
\end{equation}

\textbf{Special case:}
\begin{equation}
\frac{1}{z} = \frac{\overline{z}}{|z|^2}, \quad z \ne 0
\label{eq:reciprocal}
\end{equation}

\end{document}

